\documentclass{article}
\usepackage{xcolor}
\usepackage{listings}

% Define colors
\definecolor{keywordblue}{RGB}{0,0,128}
\definecolor{commentgreen}{RGB}{0,128,0}
\definecolor{stringred}{RGB}{163,21,21}
\definecolor{registerpurple}{RGB}{128,0,128}
\definecolor{typecyan}{RGB}{0,128,128}
\definecolor{directiveorange}{RGB}{180,100,0}
\definecolor{numberbrown}{RGB}{128,64,0}
\definecolor{labelteal}{RGB}{0,128,128}

% MicroASM language definition
\lstdefinelanguage{MicroASM}{
	morekeywords=[1]{
		% Basic Instructions
		MOV, MOVZX, MOVSX, ADD, SUB, MUL, DIV, INC, DEC,
		% Flow Control
		JMP, CMP, CALL, RET, HLT, EXIT,
		% Conditional Jumps
		JZ, JNZ, JE, JNE, JS, JNS, JC, JNC, JB, JNAE, JAE, JNB, JO, JNO,
		JG, JNLE, JL, JNGE, JGE, JNL, JLE, JNG,
		% Stack Operations
		PUSH, POP, ENTER, LEAVE,
		% I/O Operations
		IN, OUT, COUT,
		% Command Line Arguments
		ARGC, GETARG,
		% Heap Operations
		MALLOC, FREE,
		% Bitwise Operations
		AND, OR, XOR, NOT, SHL, SHR, SAR,
		% Memory Operations
		MOVADDR, MOVTO, COPY, FILL, CMP_MEM,
		% System Call
		SYSCALL,
		% Floating Point
		FMOV, FADD, FSUB, FMUL, FDIV, FCMP,
		CVTSI2SD, CVTUI2SD, CVTSD2SI, CVTSD2UI, CVTSS2SD, CVTSD2SS,
		FJE, FJNE, FJLT, FJLE, FJGT, FJGE, FJUO,
		% MNI Calls
		MNI,
		% Program Control
		LBL, STATE, MACRO, ENDMACRO, SCOPE, ENDSCOPE
	},
	morekeywords=[2]{
		% Registers
		RAX, RBX, RCX, RDX, RSI, RDI, RBP, RSP, RIP,
		R0, R1, R2, R3, R4, R5, R6, R7, R8, R9, R10, R11, R12, R13, R14, R15,
		FPR0, FPR1, FPR2, FPR3, FPR4, FPR5, FPR6, FPR7, FPR8, FPR9, FPR10, FPR11, FPR12, FPR13, FPR14, FPR15
	},
	morekeywords=[3]{
		% Data Types
		BYTE, WORD, DWORD, QWORD, FLOAT, DOUBLE, PTR
	},
	morekeywords=[4]{
		% Directives
		DB, DW, DD, DQ, DF, DDbl, RESB, RESW, RESD, RESQ, RESF, RESDbl,
		\#include
	},
	morecomment=[l]{;},
	morestring=[b]{"},
	sensitive=true
}

% Style definition for MicroASM
\lstdefinestyle{MicroASMstyle}{
	language=MicroASM,
	basicstyle=\ttfamily\small,
	keywordstyle=[1]\color{keywordblue}\bfseries,
	keywordstyle=[2]\color{registerpurple},
	keywordstyle=[3]\color{typecyan},
	keywordstyle=[4]\color{directiveorange},
	commentstyle=\color{commentgreen}\itshape,
	stringstyle=\color{stringred},
	numbers=left,
	numberstyle=\tiny\color{gray},
	stepnumber=1,
	numbersep=5pt,
	backgroundcolor=\color{white},
	showspaces=false,
	showstringspaces=false,
	showtabs=false,
	tabsize=4,
	captionpos=b,
	breaklines=true,
	breakatwhitespace=true,
	frame=single,
	rulecolor=\color{lightgray},
	escapeinside={\%*}{*)}
}

% Custom colors for specific elements
\newcommand{\microasmregister}[1]{\textcolor{registerpurple}{\texttt{#1}}}
\newcommand{\microasmkeyword}[1]{\textcolor{keywordblue}{\texttt{#1}}}
\newcommand{\microasmtype}[1]{\textcolor{typecyan}{\texttt{#1}}}
\newcommand{\microasmdirective}[1]{\textcolor{directiveorange}{\texttt{#1}}}
\newcommand{\microasmcomment}[1]{\textcolor{commentgreen}{\texttt{;#1}}}
\newcommand{\microasmstring}[1]{\textcolor{stringred}{\texttt{"#1"}}}
\newcommand{\microasmnumber}[1]{\textcolor{numberbrown}{\texttt{#1}}}
\newcommand{\microasmlabel}[1]{\textcolor{labelteal}{\texttt{#1}}}

\begin{document}
	
	% Example usage
	\begin{lstlisting}[style=MicroASMstyle]
		; Example MicroASM program with syntax highlighting
		LBL main SCOPE
		STATE counter <QWORD> 0
		STATE message <PTR>
		
		; Initialize message
		DB $1000 "Hello, MicroASM!"
		MOV message 1000
		
		; Loop 5 times
		MOV RCX 5
		LBL loop
		INC counter
		OUT 1 message
		DEC RCX
		JNZ #loop
		
		; Allocate memory
		MALLOC RAX 64
		CMP RAX 0
		JL #error
		
		; Free memory
		FREE RAX RAX
		
		HLT
		
		LBL error
		DB $2000 "Memory allocation failed!"
		OUT 2 $2000
		HLT
		ENDSCOPE
	\end{lstlisting}
	
	% Another example with floating point operations
	\begin{lstlisting}[style=MicroASMstyle]
		; Floating point example
		LBL math_demo
		STATE pi <DOUBLE> 3.14159
		STATE radius <DOUBLE> 5.0
		STATE area <DOUBLE>
		
		; Calculate area = pi * radius^2
		FMOV FPR0 pi
		FMOV FPR1 radius
		FMUL FPR1 FPR1    ; radius^2
		FMUL FPR0 FPR1    ; pi * radius^2
		FMOV area FPR0
		
		; Convert to string and output
		MNI Convert.doubleToString area R1
		OUT 1 R1
		
		HLT
	\end{lstlisting}
	
	% System call example
	\begin{lstlisting}[style=MicroASMstyle]
		; System call example
		LBL syscall_demo
		; Write to stdout
		MOV RAX 1          ; write syscall
		MOV RDI 1          ; stdout
		MOV RSI message    ; message pointer
		MOV RDX msg_len    ; message length
		SYSCALL
		
		; Exit program
		MOV RAX 60         ; exit syscall
		MOV RDI 0          ; exit code
		SYSCALL
		
		message: DB "Hello from SYSCALL!\n"
		msg_len: EQU 19
	\end{lstlisting}
	
\end{document}