\documentclass[a4paper,11pt]{book}
\usepackage[T1]{fontenc}
\usepackage[utf8]{inputenc}
\usepackage[english]{babel}
\usepackage{lmodern}
\usepackage{amsmath}
\usepackage{amsfonts}
\usepackage{amssymb}
\usepackage{amsthm}
\usepackage{graphicx}
\usepackage{color}
\usepackage{xcolor}
\usepackage{url}
\usepackage{theorem}
\usepackage{textcomp}
\usepackage{listings}
\usepackage{hyperref}
%\usepackage{glossaries}
\usepackage{parskip}
\usepackage{fontawesome}
\usepackage{fontspec}



\begin{document}

\title{Micro-Assembly}
\author{charlie-santana}
\date{\today}
\fontspec{Noto-Sans-Mono}
\maketitle
\tableofcontents
\chapter{Intro}
Welcome to the official handbook for Everything Micro-Assembly related.

This book contains information about Different contexts for Micro-Assembly
some of these contexts are
\begin{list}{*}{}
	\item What is Micro-Assembly (masm)
	\item Where it came from
	\item What is Module Native Interface
	\item MVC
	\item and much more
\end{list}

for more information about how to use Micro-Assembly, check out the \hyperref[setup:masm]{Setting up Micro-Assembly} page on how to install and create masm programs.


\text 

\subsection{Intro}
In the world of software, There's loads of programming languages to go around.
Each language fit's into a certain category, C for building systems, Java for TV remotes and android apps / minecraft mods, lua for source engine games.

it's not a surprise that everyone want's to build a programming language, weather it be interpreted or compiled, people want to build things like this either because it's fun or because they want to see how these things work.

that's why i took on the challenge to make a programming language, not a big and daunting one. just one that fit the needs of what i wanted at the time.

Since then, i've (charlie-san) been working on Micro-Assembly on and off to create something that works for everyone.

\section{history}

\subsection{what micro-assembly was}
When i built the first version of Micro-Assembly, i didn't know what i was getting into.
People told me that it would be hard to make a Assembly like programming language and not have the environment explode on me.

and for a few day's, they were right.

it felt like pain.
day after day, working on what seemed to be empty promises.

Components breaking, instances crashing.
day after day, after day of consistent punishment.

until it stopped.

one day, things got better.

the research i was doing was helping.

features got implemented faster and faster over time.

first it was the basic instructions, then it was jumps.
labels, and then \hyperref[subpar:MNI]{MNI}.

We got state for variables, Macros for reusable code and Syscalls for interacting with
the VM.

Micro-Assembly got these features in the most shortest time I've ever seen a language build.

it's incredible to see a language that only existed a single year, get these feature updates
at insane speeds.


\subsection{How it came to be}
Micro-Assembly was a joint project between me and some of my friends inside a game called
\hyperref{https://resonite.com}{}{}{Resonite}, it's main goal was to be able to build games and interactive 
worlds with a text based programming language.

Resonite has a big history behind it.
From it's initial conception as Neos to the day the owners fell apart.
it's had it's fair share of issues and triumphs.



\subsection{MASM's future}
Micro-Assembly will forever be updated with features that make it more practical and special.

For more information about what kind of features are going to be added, please check out
\hyperref{https://discord.finite.ovh}{}{}{Finite's discord server}

\subsection{Native Interface}
\subsection{General Consensus}
\subsection{MNI}\label{subpar:MNI}
\subsection{.}
\subsection{.}
\subsection{.}

\chapter{Getting started}
This chapter Will help you get started on Setting up Micro-Assembly and teach you the basics for Micro-Assembly.
\subsection{Setting up Micro-Assembly}\label{setup:masm}
\subsection{basic features}
Micro-Assembly comes with the basic features that a general porpous	
\subsection{Extended features}
\subsection{.}
\subsection{.}
\subsection{.}
\subsection{.}
\subsection{.}
\subsection{.}

\chapter{Notes}

\subsection{.}
\subsection{.}
\subsection{.}
\subsection{.}
\subsection{.}
\subsection{.}
\subsection{.}
\subsection{.}
\subsection{.}

\chapter{Basics of Micro-Assembly}

\subsection{.}
\subsection{.}
\subsection{.}
\subsection{.}
\subsection{.}
\subsection{.}
\subsection{.}
\subsection{.}
\subsection{.}

\chapter{More Advanced Topics}

\subsection{.}
\subsection{.}
\subsection{.}
\subsection{.}
\subsection{.}
\subsection{.}
\subsection{.}
\subsection{.}
\subsection{.}

\chapter{MVC}

\subsection{.}
\subsection{.}
\subsection{.}
\subsection{.}
\subsection{.}
\subsection{.}
\subsection{.}
\subsection{.}
\subsection{.}

\chapter{it's Micro-High, not UHigh.net}

Micro-High is a C like language designed to be a Easy to use programming language for the 
Micro-Assembly Ecoystem

\subsection{.}
\subsection{.}
\subsection{.}
\subsection{.}
\subsection{.}
\subsection{.}
\subsection{.}
\subsection{.}
\subsection{.}

\chapter{Micro-High Notes}

\subsection{.}
\subsection{.}
\subsection{.}
\subsection{.}
\subsection{.}
\subsection{.}
\subsection{.}
\subsection{.}
\subsection{.}

\end{document}