\documentclass[a4paper,11pt]{book}
\usepackage[T1]{fontenc}
\usepackage[utf8]{inputenc}
\usepackage[english]{babel}
\usepackage{lmodern}
\usepackage{amsmath}
\usepackage{amsfonts}
\usepackage{amssymb}
\usepackage{amsthm}
\usepackage{graphicx}
\usepackage{color}
\usepackage{xcolor}
\usepackage{url}
\usepackage{theorem}
\usepackage{textcomp}
\usepackage{listings}
\usepackage{tikz}

\usepackage[colorlinks=true, linkcolor=commentgreen, urlcolor=cyan, citecolor=green]{hyperref}

%\usepackage{glossaries}
\usepackage{parskip}

\usepackage{fontspec}
% Mono font setup (use a TeX Live-safe default)
\defaultfontfeatures{Scale=MatchLowercase}
\setmonofont{Latin Modern Mono}
\fontspec{Noto-Sans-Mono}
% Define colors
\definecolor{keywordblue}{RGB}{0,0,128}
\definecolor{commentgreen}{RGB}{0,128,0}
\definecolor{stringred}{RGB}{163,21,21}
\definecolor{registerpurple}{RGB}{128,0,128}
\definecolor{typecyan}{RGB}{0,128,128}
\definecolor{directiveorange}{RGB}{180,100,0}
\definecolor{numberbrown}{RGB}{128,64,0}
\definecolor{labelteal}{RGB}{0,128,128}
\definecolor{microblue}{RGB}{30, 120, 200}
% MicroASM language definition
\lstdefinelanguage{MicroASM}{
	morekeywords=[1]{
		% Basic Instructions
		MOV, MOVZX, MOVSX, ADD, SUB, MUL, DIV, INC, DEC,
		% Flow Control
		JMP, CMP, CALL, RET, HLT, EXIT,
		% Conditional Jumps
		JZ, JNZ, JE, JNE, JS, JNS, JC, JNC, JB, JNAE, JAE, JNB, JO, JNO,
		JG, JNLE, JL, JNGE, JGE, JNL, JLE, JNG,
		% Stack Operations
		PUSH, POP, ENTER, LEAVE,
		% I/O Operations
		IN, OUT, COUT,
		% Command Line Arguments
		ARGC, GETARG,
		% Heap Operations
		MALLOC, FREE,
		% Bitwise Operations
		AND, OR, XOR, NOT, SHL, SHR, SAR,
		% Memory Operations
		MOVADDR, MOVTO, COPY, FILL, CMP_MEM,
		% System Call
		SYSCALL,
		% Floating Point
		FMOV, FADD, FSUB, FMUL, FDIV, FCMP,
		CVTSI2SD, CVTUI2SD, CVTSD2SI, CVTSD2UI, CVTSS2SD, CVTSD2SS,
		FJE, FJNE, FJLT, FJLE, FJGT, FJGE, FJUO,
		% MNI Calls
		MNI,
		% Program Control
		LBL, STATE, MACRO, ENDMACRO, SCOPE, ENDSCOPE
	},
	morekeywords=[2]{
		% Registers
		RAX, RBX, RCX, RDX, RSI, RDI, RBP, RSP, RIP,
		R0, R1, R2, R3, R4, R5, R6, R7, R8, R9, R10, R11, R12, R13, R14, R15,
		FPR0, FPR1, FPR2, FPR3, FPR4, FPR5, FPR6, FPR7, FPR8, FPR9, FPR10, FPR11, FPR12, FPR13, FPR14, FPR15
	},
	morekeywords=[3]{
		% Data Types
		BYTE, WORD, DWORD, QWORD, FLOAT, DOUBLE, PTR
	},
	morekeywords=[4]{
		% Directives
		DB, DW, DD, DQ, DF, DDbl, RESB, RESW, RESD, RESQ, RESF, RESDbl,
		\#include
	},
	morecomment=[l]{;},
	morestring=[b]{"},
	sensitive=true
}

% Style definition for MicroASM
\lstdefinestyle{MicroASMstyle}{
	language=MicroASM,
	basicstyle=\ttfamily\small,
	keywordstyle=[1]\color{keywordblue}\bfseries,
	keywordstyle=[2]\color{registerpurple},
	keywordstyle=[3]\color{typecyan},
	keywordstyle=[4]\color{directiveorange},
	commentstyle=\color{commentgreen}\itshape,
	stringstyle=\color{stringred},
	numbers=left,
	numberstyle=\tiny\color{black},
	stepnumber=1,
	numbersep=5pt,
	backgroundcolor=\color{white},
	showspaces=false,
	showstringspaces=false,
	showtabs=false,
	tabsize=4,
	captionpos=b,
	breaklines=true,
	breakatwhitespace=true,
	frame=single,
	rulecolor=\color{lightgray},
	escapeinside={\%*}{*)}
}

% Custom colors for specific elements
\newcommand{\microasmregister}[1]{\textcolor{registerpurple}{\texttt{#1}}}
\newcommand{\microasmkeyword}[1]{\textcolor{keywordblue}{\texttt{#1}}}
\newcommand{\microasmtype}[1]{\textcolor{typecyan}{\texttt{#1}}}
\newcommand{\microasmdirective}[1]{\textcolor{directiveorange}{\texttt{#1}}}
\newcommand{\microasmcomment}[1]{\textcolor{commentgreen}{\texttt{;#1}}}
\newcommand{\microasmstring}[1]{\textcolor{stringred}{\texttt{"#1"}}}
\newcommand{\microasmnumber}[1]{\textcolor{numberbrown}{\texttt{#1}}}
\newcommand{\microasmlabel}[1]{\textcolor{labelteal}{\texttt{#1}}}


\begin{document}

\title{Micro-Assembly}
\author{charlie-santana}
\date{\today}

\begin{titlepage}
	\centering
	\vspace*{3cm}
	
	% --- Stylized "logo" title ---
	{\Huge\bfseries\textsf{\color{microblue} Micro}\color{black}-Assembly}\par
	\vspace{0.6cm}
	{\Large\itshape The Official Handbook for Everything MicroASM}\par
	
	\vspace{2.5cm}
	
	% --- Decorative separator ---
	\rule{0.8\textwidth}{0.5pt}\par
	\vspace{0.4cm}
	{\large A \textbf{Finite Organization} Project}\par
	\rule{0.8\textwidth}{0.5pt}\par
	
	\vspace{2cm}
	
	% --- Metadata ---
	{\large Written by charlie-santana}\par
	\vspace{0.2cm}
	{\small Version 1.0 — Compiled on \today}\par
	
	\vfill
	
	% --- Footer ---
	{\footnotesize Open-source Documentation \textbullet\ Licensed under CC BY-SA}\par
\end{titlepage}


\tableofcontents
\chapter{Intro}
Welcome to the official handbook for Everything Micro-Assembly related.

This book contains information about Different contexts for Micro-Assembly
some of these contexts are
\begin{list}{*}{}
	\item What is Micro-Assembly (masm)
	\item Where it came from
	\item What is Module Native Interface
	\item MVC
	\item and much more
\end{list}

for more information about how to use Micro-Assembly, check out the \hyperref[setup:masm]{Setting up Micro-Assembly} page on how to install and create masm programs.


\text 


\subsection{Intro}
In the world of software, There's loads of programming languages to go around.
Each language fit's into a certain category, C for building systems, Java for TV remotes and android apps / minecraft mods, lua for source engine games.

it's not a surprise that everyone want's to build a programming language, weather it be interpreted or compiled, people want to build things like this either because it's fun or because they want to see how these things work.

that's why i took on the challenge to make a programming language, not a big and daunting one. just one that fit the needs of what i wanted at the time.

Since then, i've (charlie-san) been working on Micro-Assembly on and off to create something that works for everyone.

\section{history}

\subsection{what micro-assembly was}
When i built the first version of Micro-Assembly, i didn't know what i was getting into.
People told me that it would be hard to make a Assembly like programming language and not have the environment explode on me.

and for a few day's, they were right.

it felt like pain.
day after day, working on what seemed to be empty promises.

Components breaking, instances crashing.
day after day, after day of consistent punishment.

until it stopped.

one day, things got better.

the research i was doing was helping.

features got implemented faster and faster over time.

first it was the basic instructions, then it was jumps.
labels, and then \hyperref[subpar:MNI]{MNI}.

We got state for variables, Macros for reusable code and Syscalls for interacting with
the VM.

Micro-Assembly got these features in the most shortest time I've ever seen a language build.

it's incredible to see a language that only existed a single year, get these feature updates
at insane speeds.


\subsection{How it came to be}
Micro-Assembly was a joint project between me and some of my friends inside a game called
\hyperref{https://resonite.com}{}{}{Resonite}, it's main goal was to be able to build games and interactive 
worlds with a text based programming language.

Developing items in resonite requires the use of their custom scripting language called ProtoFlux.

Protoflux is a Visual scripting language following the typical left to right layout of most programming languages..
It's main features do not include defining array's, Regex parsing or the likes.

it's feature set does contain string manipulation and the very basics to get you started,
though it's more in line with making game's with it instead of proper programming languages.
it's more compared to C then it is an actual dot-net language.

Though, with it's challenges and quirks, it is more than possible (if not just hard) to make a programming language with Protoflux.

The original idea on how Micro-Assembly started was because i was talking to one of my friends who is really into old computers.

at the time, they were trying to setup a model for one of those big room scale computer's.
the old ones like ibm 350 and the likes.
i saw the machines they were trying to setup and an idea came to me for wanting to make something that could
work with these machines, like being able to program them with actual software and have them do something.

though that idea took a bit to get started, the first prototype took a good few days to implement.
we pretty much went through all of the different ways to do something like this. 

once we found the working way to do a proper assembly like interpreter, it only really took like half an hour to implement the basic instructions like Add, Mul, Sub, Div, basic jump operators and things like that.

labels of course came way into the future.

\subsection{MASM's future}
Micro-Assembly will forever be updated with features that make it more practical and special.

For more information about what kind of features are going to be added, please check out
\hyperref{https://discord.finite.ovh}{}{}{Finite's discord server}


\subsection{General Consensus}
\subsection{.}
\subsection{.}
\subsection{.}

\chapter{Getting started}
This chapter Will help you get started on Setting up Micro-Assembly and teach you the basics for Micro-Assembly.
\subsection{Setting up Micro-Assembly}\label{setup:masm}
\subsection{basic features}
Micro-Assembly comes with the basic features that a general porpous	
\subsection{Extended features}
\subsection{MNI}\label{subpar:MNI}
MNI or the Module native interface allows developers to design Foren functions
\subsection{.}
\subsection{.}
\subsection{.}
\subsection{.}
\subsection{.}

\chapter{Notes}

\subsection{.}
\subsection{.}
\subsection{.}
\subsection{.}
\subsection{.}
\subsection{.}
\subsection{.}
\subsection{.}
\subsection{.}

\chapter{Basics of Micro-Assembly}

\subsection{.}
\subsection{.}
\subsection{.}
\subsection{.}
\subsection{.}
\subsection{.}
\subsection{.}
\subsection{.}
\subsection{.}

\chapter{More Advanced Topics}

\subsection{.}
\subsection{.}
\subsection{.}
\subsection{.}
\subsection{.}
\subsection{.}
\subsection{.}
\subsection{.}
\subsection{.}

\chapter{MVC}

\subsection{.}
\subsection{.}
\subsection{.}
\subsection{.}
\subsection{.}
\subsection{.}
\subsection{.}
\subsection{.}
\subsection{.}

\chapter{it's Micro-High, not UHigh.net}

Micro-High is a C like language designed to be a Easy to use programming language for the 
Micro-Assembly Ecoystem

\subsection{.}
\subsection{.}
\subsection{.}
\subsection{.}
\subsection{.}
\subsection{.}
\subsection{.}
\subsection{.}
\subsection{.}

\chapter{Micro-High Notes}

\subsection{.}
\subsection{.}
\subsection{.}
\subsection{.}
\subsection{.}
\subsection{.}
\subsection{.}
\subsection{.}
\subsection{.}

\end{document}