\documentclass[11pt]{article}
\usepackage[utf8]{inputenc}
\usepackage[T1]{fontenc}
\usepackage{hyperref}
\usepackage{geometry}
\usepackage{graphicx}
\usepackage{xcolor}
\usepackage{listings}
\usepackage{amsmath}
\usepackage{amssymb}
\usepackage{booktabs}
\usepackage{tabularx}
\usepackage{enumitem}
\usepackage{titlesec}
\usepackage{fancyhdr}
\usepackage{lastpage}

% Page layout
\geometry{a4paper, margin=1in}
\setlength{\parindent}{0pt}
\setlength{\parskip}{6pt}

% Colors
\definecolor{codegray}{rgb}{0.5,0.5,0.5}
\definecolor{backcolour}{rgb}{0.95,0.95,0.92}

% Listings style
\lstset{
	backgroundcolor=\color{backcolour},
	commentstyle=\color{green},
	keywordstyle=\color{blue},
	numberstyle=\tiny\color{codegray},
	stringstyle=\color{red},
	basicstyle=\ttfamily\footnotesize,
	breakatwhitespace=false,
	breaklines=true,
	captionpos=b,
	keepspaces=true,
	numbers=left,
	numbersep=5pt,
	showspaces=false,
	showstringspaces=false,
	showtabs=false,
	tabsize=2
}

% Header and footer
\pagestyle{fancy}
\fancyhf{}
\fancyhead[L]{\textit{\proposalTitle}}
\fancyhead[R]{\proposalType\ \proposalNumber}
\fancyfoot[C]{\thepage\ of \pageref{LastPage}}

% Title formatting
\titleformat{\section}
{\Large\bfseries}
{\thesection}{1em}{}
\titleformat{\subsection}
{\large\bfseries}
{\thesubsection}{1em}{}
\titleformat{\subsubsection}
{\bfseries}
{\thesubsubsection}{1em}{}

% Custom commands
\newcommand{\proposalType}{MEP}
\newcommand{\proposalNumber}{XXXX}
\newcommand{\proposalTitle}{Proposal Title}
\newcommand{\proposalStatus}{Draft}
\newcommand{\proposalVersion}{0.1}
\newcommand{\proposalDate}{\today}
\newcommand{\proposalAuthor}{Author Name}
\newcommand{\proposalAuthorsEmail}{author@example.com}
\newcommand{\proposalSponsor}{Sponsor Name (if any)}
\newcommand{\proposalDiscussion}{Discussion URL}

\begin{document}
	
	% Title page
	\begin{titlepage}
		\centering
		
		\vspace*{2cm}
		
		{\Huge\bfseries \proposalType-\proposalNumber\\}
		\vspace{0.5cm}
		{\LARGE \proposalTitle}
		
		\vspace{2cm}
		
		\begin{tabular}{ll}
			\textbf{Title:} & \proposalTitle \\
			\textbf{Status:} & \proposalStatus \\
			\textbf{Type:} & Standards Track / Informational / Process \\
			\textbf{Version:} & \proposalVersion \\
			\textbf{Date:} & \proposalDate \\
			\textbf{Author:} & \proposalAuthor \\
			\textbf{Email:} & \proposalAuthorsEmail \\
			\textbf{Sponsor:} & \proposalSponsor \\
			\textbf{Discussion:} & \proposalDiscussion \\
		\end{tabular}
		
		\vfill
		
		\begin{abstract}
			This enhancement proposal describes [brief summary of what the proposal does and why it's important]. 
			It addresses [key problem or opportunity] by [main approach]. 
			The proposed changes will [expected benefits or outcomes]. 
			This proposal is relevant for [target audience or components affected].
		\end{abstract}
		
		\vspace{1cm}
		
		\rule{\textwidth}{0.4pt}
		
		\vspace{0.5cm}
		
		\small
		Copyright \textcopyright{} \the\year{} \proposalAuthor
		
		This work is licensed under a Creative Commons Attribution 4.0 International License.
		
	\end{titlepage}
	
	% Table of contents
	\tableofcontents
	\newpage
	
	% Main content
	\section{Introduction}
	This section provides an overview of the enhancement proposal, including the context and background information necessary to understand the proposal. It should introduce the problem space and give a high-level description of the proposed solution.
	
	\subsection{Background}
	Relevant background information that helps readers understand the context of this proposal.
	
	\subsection{Scope}
	What is included in this proposal and, just as importantly, what is excluded.
	
	\section{Motivation}
	Describe why this change is being proposed. What problem does it solve? What use cases does it address? Why is this change important now?
	
	Key motivations may include:
	\begin{itemize}
		\item Performance improvements
		\item Usability enhancements
		\item Security considerations
		\item Maintenance and technical debt reduction
		\item New feature requirements
		\item Community or user requests
	\end{itemize}
	
	\section{Rationale}
	Explain the reasoning behind the design decisions. Why was this particular approach chosen over others? Discuss the trade-offs considered and the principles that guided the design.
	
	\subsection{Design Principles}
	\begin{itemize}
		\item Principle 1 and how it applies
		\item Principle 2 and how it applies
		\item Principle 3 and how it applies
	\end{itemize}
	
	\subsection{Trade-offs}
	Discussion of trade-offs considered during the design process.
	
	\section{Specification}
	The technical specification of the proposal. This should be detailed enough to allow implementation.
	
	\subsection{Technical Details}
	Describe the technical aspects here, including any algorithms, data structures, or architectural changes.
	
	\subsection{API Changes}
	List any API changes if applicable. Use code examples where helpful:
	
	\begin{lstlisting}[language=Java,caption=Example API change]
		// Old API
		public void oldMethod(String param);
		
		// New API
		public void newMethod(String param, Options options);
	\end{lstlisting}
	
	\subsection{Syntax Changes}
	Describe syntax changes if any. For language changes, provide before and after examples.
	
	\subsection{Behavioral Changes}
	Describe how behavior changes with this proposal.
	
	\section{Backwards Compatibility}
	Discuss backwards compatibility implications and migration strategies if applicable.
	
	\subsection{Breaking Changes}
	List any breaking changes and their impact.
	
	\subsection{Migration Path}
	Describe how existing code can be migrated to work with the new changes.
	
	\subsection{Deprecation Strategy}
	If applicable, describe the deprecation timeline and process.
	
	\section{Security Implications}
	Describe any security implications of this proposal. Consider:
	
	\begin{itemize}
		\item New attack vectors introduced
		\item Security improvements
		\item Authentication/authorization changes
		\item Data protection implications
	\end{itemize}
	
	\section{Performance Implications}
	Discuss performance implications and provide benchmarks if available.
	
	\subsection{Performance Improvements}
	Describe expected performance improvements.
	
	\subsection{Performance Costs}
	Describe any performance costs or trade-offs.
	
	\subsection{Benchmarks}
	If available, include benchmark results comparing old and new approaches.
	
	\section{Alternatives}
	Discuss alternative approaches that were considered and why they were rejected.
	
	\subsection{Alternative Approach 1}
	Description of alternative and reasons for rejection.
	
	\subsection{Alternative Approach 2}
	Description of alternative and reasons for rejection.
	
	\subsection{Alternative Approach 3}
	Description of alternative and reasons for rejection.
	
	\section{Reference Implementation}
	Describe or reference the implementation. This section is optional for some proposals but can be helpful for reviewers.
	
	\subsection{Implementation Details}
	Key details about the implementation.
	
	\subsection{Testing Strategy}
	How the implementation was tested.
	
	\section{Test Cases}
	Provide test cases to verify the implementation.
	
	\subsection{Unit Tests}
	Description of unit tests.
	
	\subsection{Integration Tests}
	Description of integration tests.
	
	\subsection{Example Usage}
	Practical examples of how to use the new feature:
	
	\begin{lstlisting}[language=Python,caption=Example usage]
		def example_usage():
		# Before the change
		result = old_function(data)
		
		# After the change
		result = new_function(data, improved=True)
		return result
	\end{lstlisting}
	
	\section{References}
	\begin{enumerate}
		\item Related PEP/JEP/RFC documents
		\item Technical papers or articles
		\item Source code references
		\item Discussion threads
	\end{enumerate}
	
	\section{Acknowledgments}
	Acknowledge contributors, reviewers, and anyone who helped with the proposal. Be generous with credit where it's due.
	
	\appendix
	\section{Appendix A: Additional Information}
	Any additional material that supports the proposal but isn't essential to the main text.
	
	\subsection{Detailed Technical Analysis}
	More in-depth technical analysis if needed.
	
	\subsection{Additional Examples}
	More examples of usage patterns.
	
	\subsection{Implementation Timeline}
	If applicable, provide a proposed implementation timeline.
	
	\section{Appendix B: Revision History}
	\begin{itemize}
		\item Version 0.1 (YYYY-MM-DD): Initial draft
		\item Version 0.2 (YYYY-MM-DD): Incorporated feedback from discussion
		\item Version 1.0 (YYYY-MM-DD): Final version accepted
	\end{itemize}
	
\end{document}